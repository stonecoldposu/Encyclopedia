% If the calculus has an acronym, define it.
% (e.g. \newcommand{\LK}{\ensuremath{\mathbf{LK}}\xspace})

\calculusName{IR}   % The name of the calculus
\calculusAcronym{IR}    % The acronym if defined above, or empty otherwise. 
\calculusLogic{False  Quantified Boolean Formulas in Closed Prenex CNF}  % Specify the logic (e.g. classical, intuitionistic, ...) for which this calculus is intended.
\calculusType{Resolution}   % Specify the calculus type (e.g. Frege-Hilbert style, tableau, sequent calculus, hypersequent calculus, natural deduction, ...)
\calculusYear{2014}   % The year when the calculus was invented.
\calculusAuthor{Olaf Beyersdorff, Leroy Chew, Mikol\'{a}\v{s} Janota } % The name(s) of the author(s) of the calculus.


\entryTitle{IR}     % Title of the entry (usually coincides with the name of the calculus).
\entryAuthor{Leroy Chew \and Mikol\'{a}\v{s} Janota}    % Your name(s). Separate multiple names with "\and".
\maketitle

\tag{Resolution}
\tag{Refutation}
\tag{Instantiation}
\tag{CNF}
\tag{Closed Prenex QBF}
\tag{QBF}

% If you wish, use tags to give any other information 
% that might be helpful for classifying and grouping this entry:
% e.g. \tag{Two-Sided Sequents}
% e.g. \tag{Multiset Cedents}
% e.g. \tag{List Cedents}
% You are free to invent your own tags. 
% The Encyclopedia's coordinator will take care of 
% merging semantically similar tags in the future.


%\maketitle


% If your files are called "<ID>.tex" and "<ID>.bib", 
% then you should write "\begin{entry}{<ID>}" in the line below
\begin{entry}{IR}  

% Define here any newcommands you may need:
% e.g. \newcommand{\necessarily}{\Box}
% e.g. \newcommand{\possibly}{\Diamond}
%\DeclareMathOperator*{\lev}{\operatorname{lv}}
%\DeclareMathOperator*{\instantiate}{\textsf{inst}}
%\DeclareMathOperator*{\range}{{\sf rng}}
\newcommand{\comprehension}[2]{\ensuremath{\left\{ {#1} \;|\; {#2}\right\}}}
\newcommand{\lev}{{\sf lv}}
\newcommand{\restr}{{\sf restr}}
\newcommand{\range}{{\sf rng}}
\newcommand{\domain}{{\sf dom}}

\begin{calculus}

% Add the inference rules of your proof system here.
% The "proof.sty" and "bussproofs.sty" packages are available.
% If you need any other package, please contact the editor (bruno@logic.at)
%\framebox{
%  \parbox{0.4\breite}
%  {\small
    \begin{prooftree}
      \AxiomC{}
      \RightLabel{(Ax)}
      \UnaryInfC{
        $\comprehension{x^{\hspace{1pt}\restr(\tau,x)}}
                       {x\in C, x\textrm{ is existential}} $}
    \end{prooftree}
    $C$ is a non-tautological clause from the matrix. \\$\tau=\comprehension{0/u}{u\textrm{ is universal in }C}$, where the notation $0/u$ for literals $u$ is shorthand for $0/y$ if $u=y$ and $1/y$ if $u=\neg y$. We define $\restr(\tau, x)$ as ${\comprehension{c/u}{c/u\in \tau, \lev(u)<\lev(x)}}$.
    \begin{prooftree}
      \AxiomC{$x^\tau\lor C_1 $ }
      \AxiomC{$\lnot x^\tau\lor C_2 $}
      \RightLabel{(Resolution)}
      \BinaryInfC{$C_1\cup C_2$}
 \end{prooftree}
    \begin{prooftree}
      \AxiomC{$C$}
      \RightLabel{(Instantiation)}
      \UnaryInfC{$\comprehension{x^{\hspace{1pt}\xi}}{x^{\hspace{1pt}\sigma}\in C, x\textrm{ is existential}}$}
    \end{prooftree}
    $\tau$ is a partial assignment to universal variables with $\range(\tau) \subseteq \{0,1\}$.
    $\xi=\sigma\cup\comprehension{c/u}{c/u \in \restr(\tau,x), u \notin \domain(\sigma)}$
  %\caption{The rules of \irc  \cite{MFCS14}}

    \centerline{The rules of IR~\cite{MFCS14}}
\end{calculus}


\begin{clarifications}
  The calculus aims to refute a quantified Boolean formula (QBF) of the form
  $Q_1 x_1\dots Q_n x_n.\,\phi$ where $Q_i\in\{\forall,\exists\}$
  and $\phi$ is a Boolean formula in conjunctive normal form (CNF).
  The formula $\phi$ is referred to as the \emph{matrix}.
  We write $\lev(x)$ for the \emph{quantification level} of~$x$, i.e.\ $\lev(x_i)=i$.
  A variable~$x_i$ is \emph{existential} (resp.\ \emph{universal}) if $Q_i=\exists$ (resp.\ $Q_i=\forall$).

  The calculus works by introducing clauses as \emph{annotated clauses}, which are sets of annotated literals. Annotated literals consist of an existential literal and an annotation -- a partial assignment to universal variables in $\{0,1\}$. Two literals are identical if and only if both the existential literal and annotation are equal.
%
  The calculus enables deriving the empty clause if and only if the given formula is false.
\end{clarifications}


\begin{technicalities}
Soundness was shown by extracting valid Herbrand functions. Completeness is shown by p-simulation of 
another known QBF system Q-Resolution \irefmissing{QResolution}.
\end{technicalities}


% The following environments ("clarifications", "history", 
% "technicalities") are optional. If you do use them, 
% be very concise and objective.

% \begin{clarifications}
% ToDo: write here short remarks that may help the reader to understand 
% the inference rules of the proof system.
% \end{clarifications}

\begin{history}
The name of the calculus comes from the two pivotal operations \emph{instantiation} and \emph{resolution}.
The calculus naturally generalizes an older calculus $\forall$Exp+Res~\cite{JanotaTCS15},
which  requires all clauses to be introduced into the proof by using a complete assignment.
\irefmissing{SuggestedIDForOtherProofSystem}
% ToDo: write here short historical remarks about this proof system,
% especially if they relate to other proof systems. 
% Use "\iref{OtherProofSystem}" to refer to another proof system 
% in the Encyclopedia (where "OtherProofSystem" is its ID). 
% Use "\irefmissing{SuggestedIDForOtherProofSystem}" to refer to 
% another proof system that is not yet available in the encyclopedia.
\end{history}

% \begin{technicalities}
% ToDo: write here remarks about soundness, completeness, decidability...
% \end{technicalities}



% Please cite the original paper where the proof system was defined.
% To do so, you may use the \cite command within 
% one of the optional environments above,
% or use the \nocite command otherwise.

% You may also cite a modern paper or book where the 
% proof system is explained in greater depth or clarity.
% Cite parsimoniously.

% Do not cite related work. Instead, use the "\iref" or "\irefmissing" 
% commands to make an internal reference to another entry, 
% as explained within the "history" environment above.

% You do not need to create the "References" section yourself. 
% This is done automatically.


% Leave an empty line above "\end{entry}".

\end{entry}
